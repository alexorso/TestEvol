\documentclass[conference]{IEEEtran}
\synctex=1

\usepackage[linesnumbered,boxed,ruled,nofillcomment]{algorithm2e} % add vlined,
\usepackage{amsmath}
%\usepackage{amsthm}
\usepackage[usenames]{color}
\usepackage[light,first]{draftcopy}
% \usepackage{draftwatermark}
% \usepackage[firstpage]{draftwatermark}
\usepackage{epsfig}
%\usepackage{fullpage}
%\usepackage[pdftex]{graphicx}
\usepackage{graphicx}
%\usepackage{latex8}
\usepackage{listings}
\lstloadlanguages{Java}
\usepackage{multirow}
%\usepackage{palatino}
\usepackage{alltt}
\usepackage{times}
% \usepackage{url}
\usepackage{hyperref}
\usepackage{cite}
\usepackage{xspace}
\usepackage[linesnumbered,boxed,ruled,nofillcomment]{algorithm2e} % add vlined,
% \usepackage[small,compact]{titlesec}

\newcommand{\subparagraph}{}
%\usepackage[compact]{titlesec}
%\titlespacing{\section}{0pt}{1ex}{1ex}
%\titlespacing{\subsection}{0pt}{1ex}{0ex}
%\titlespacing{\subsubsection}{0pt}{0.5ex}{0ex}

% Use the following to make modification
% \SetWatermarkAngle{45}
% \SetWatermarkLightness{0.8}
% \SetWatermarkFontSize{5cm}
% \SetWatermarkScale{1}
% \SetWatermarkText{DRAFT -- Do not redistribute}

%\renewcommand{\topfraction}{0.85}
%\renewcommand{\textfraction}{0.1}
%\renewcommand{\floatpagefraction}{0.75}
%\addtolength{\textfloatsep}{-0.3in}

%\setpapersize{USletter}
%\setmarginsrb{2.54cm}{2.54cm}{2.54cm}{1.5cm}{0pt}{0mm}{0pt}{8mm}
%\setcounter{secnumdepth}{3}

\newcommand{\ie}{\textit{i.e.,} }
\newcommand{\eg}{\textit{e.g.,} }

\newcommand\opt[1]{}
\newcommand\find[1]{}

\newcommand{\xxx}[1]{{\color{red}\bf #1}}

\def\denseitems{
  \itemsep1pt plus1pt minus1pt
  \parsep0pt plus0pt
  \parskip0pt
  \topsep0pt
}
\leftmargini 1em
%  \leftmarginii-1em
%  \leftmarginiii-1em
%  \leftmarginvi-1em

\newenvironment{myquote}
  {\baselineskip=\SingleLine \begin{quote}
    {\setlength{\leftmargin}{2\parindent}}}
  {\end{quote} \baselineskip=\DoubleLine \vspace*{-\SingleLine}}

\def\bfcaption#1{\caption[#1]{\bf #1}}

\def\argmax{\mathop{\rm arg\,max}}

\def\denseitems{
  \itemsep1pt plus1pt minus1pt
  \parsep0pt plus0pt
  \parskip0pt\topsep0pt}

\newcommand{\ls}[1]
   {\dimen0=\fontdimen6\the\font 
    \lineskip=#1\dimen0
    \advance\lineskip.5\fontdimen5\the\font
    \advance\lineskip-\dimen0
    \lineskiplimit=.9\lineskip
    \baselineskip=\lineskip
    \advance\baselineskip\dimen0
    \normallineskip\lineskip
    \normallineskiplimit\lineskiplimit
    \normalbaselineskip\baselineskip
    \ignorespaces
   }

\newcommand{\crunchlist}{
        \setlength{\itemsep}{-0.05in}
        \setlength{\partopsep}{0in}
        \setlength{\topsep}{0in}}

\def\sqzhuge{\vspace{-14pt}}
\def\sqztiny{\vspace{-6pt}}

\def\definition#1#2{
\begin{smalldescription}
\item[Definition:] \underline{#1}\\
#2\hfill~$\Box$
\end{smalldescription}
%\vspace*{-5mm}
}

\def\definition#1{
\begin{smalldescription}
\item[Definition:]
#1\hfill~$\Box$
\end{smalldescription}
}

\def\ind#1{{\it #1}\index{#1}}

%% Alternative to itemize
\newenvironment{smallitemize}{
   \setlength{\topsep}{0pt}
   \setlength{\partopsep}{0pt}
   \setlength{\parskip}{0pt}
   \begin{itemize}
   \setlength{\leftmargin}{.2in}
   \setlength{\parsep}{0pt}
   \setlength{\parskip}{0pt}
   \setlength{\itemsep}{0pt}}{\end{itemize}}

%% Alternative to enumerate
\newenvironment{smallenumerate}{
   \setlength{\topsep}{0pt}
   \setlength{\partopsep}{0pt}
   \setlength{\parskip}{0pt}
   \begin{enumerate}
   \setlength{\leftmargin}{.2in}
   \setlength{\parsep}{0pt}
   \setlength{\parskip}{0pt}
   \setlength{\itemsep}{0pt}}{\end{enumerate}}

%% Alternative to description
\newenvironment{smalldescription}{
   \setlength{\topsep}{0pt}
   \setlength{\partopsep}{0pt}
   \setlength{\parskip}{0pt}
   \begin{description}
   \setlength{\leftmargin}{.2in}
   \setlength{\parsep}{0pt}
   \setlength{\parskip}{0pt}
   \setlength{\itemsep}{0pt}}{\end{description}}

%\long\gdef\TabPerf#1#2{%
%  #1 & #2 \\ \hline
%}

%% The [gray]{0} gives rise to the black color. For other shades of
%% gray, increase the number.
%\long\gdef\TabHead#1#2{%
%\multicolumn{1}{|>{\columncolor[gray]{0}}c|}{\textcolor{white}{\bf #1}} &
%\multicolumn{1}{>{\columncolor[gray]{0}}c|}{\textcolor{white}{\bf #2}} \\ \hline 
%}

%\long\gdef\TabPerf#1#2{%
%  #1 & #2 \\ \hline
%}

%% The [gray]{0} gives rise to the black color. For other shades of
%% gray, increase the number.
%\long\gdef\TabHead#1#2{%
%\multicolumn{1}{|>{\columncolor[gray]{0}}c|}{\textcolor{white}{\bf #1}} &
%\multicolumn{1}{>{\columncolor[gray]{0}}c|}{\textcolor{white}{\bf #2}} \\ \hline 
%}

%%% Local Variables: 
%%% mode: latex
%%% TeX-master: "paper"
%%% End: 

\newcommand{\tool}{\textsc{TestEvol}\xspace}

%\linespread{0.97}
%\columnsep 0.11in
\clubpenalty = 10000
\widowpenalty = 10000
\displaywidowpenalty = 10000

\let\oldthebibliography=\thebibliography
\let\endoldthebibliography=\endthebibliography
\renewenvironment{thebibliography}[1]{%
  \begin{oldthebibliography}{#1}%
    \setlength{\parskip}{0ex}%
    \setlength{\itemsep}{0ex}%
  }%
  {%
  \end{oldthebibliography}%
}

%\toappear{}

\begin{document}

\title{TestEvol: A Tool for Analyzing\\Test-Suite Evolution}

\author{\IEEEauthorblockN{XXX}
\IEEEauthorblockA{
College of Computing\\
Georgia Institute of Technology\\
Atlanta, USA\\
\{orso\}@cc.gatech.edu}
}

\maketitle

\pagestyle{empty}

\begin{abstract}
  Test suites, once created, rarely remain static. Just like the
  application they are testing, they evolve throughout their lifetime.
  Test obsolescence is probably the most known reason for test-suite
  evolution---test cases cease to work because of changes in the code
  and must be suitably repaired. Repairing existing test cases
  manually, however, can be extremely time consuming, especially for
  large test suites, which has motivated the recent development of
  automated test-repair techniques. We believe that, for developing
  effective repair techniques that are applicable in real-world
  scenarios, a fundamental prerequisite is a thorough understanding of
  how test cases evolve in practice. Without such knowledge, we risk
  to develop techniques that may work well for only a small number of
  tests or, worse, that may not work at all in most realistic cases.
  Unfortunately, to date there are no studies in the literature that
  investigate how test suites evolve. To tackle this problem, in this
  paper we present a technique for studying test-suite evolution, a
  tool that implements the technique, and an extensive empirical study
  in which we used our technique to study many versions of six
  real-world programs and their unit test suites. This is the first
  study of this kind, and our results reveal several interesting
  aspects of test-suite evolution. In particular, our findings show
  that test repair is just one possible reason for test-suite
  evolution, whereas most changes involve refactorings, deletions, and
  additions of test cases.  Our results also show that test
  modifications tend to involve complex, and hard-to-automate, changes
  to test cases, and that existing test-repair techniques that focus
  exclusively on assertions may have limited practical applicability.
  More generally, our findings provide initial insight on how test
  cases are added, removed, and modified in practice, and can guide
  future research efforts in the area of test-suite evolution.
\end{abstract}

% A category with the (minimum) three required fields
%\category{H.4}{Information Systems Applications}{Miscellaneous}
%A category including the fourth, optional field follows...
%\category{D.2.8}{Software Engineering}{Metrics}[complexity measures, performance measures]

%\terms{Theory}

%\keywords{ACM proceedings, \LaTeX, text tagging}

\section{Introduction}

\vspace{4pt}
The rest of the paper is organized as follows.  

\section{Conclusion and future work}
\label{sec:concl-future-work}


\section*{Acknowledgments}

This work was supported in part by NSF awards CCF-0916605 and
CCF-0964647 to Georgia Tech, and by funding from IBM Research and
Microsoft Research.

\bibliographystyle{abbrv}
\bibliography{paper}

\end{document}

