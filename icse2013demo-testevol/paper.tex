\documentclass[conference]{IEEEtran}
%\synctex=1

\usepackage{times}
\usepackage{graphicx}
\usepackage{epsfig}
% \usepackage{url}
% \usepackage{hyperref}
\usepackage{hyperref}
\usepackage{xspace}\usepackage{float}
\usepackage{latexsym}
\usepackage{natbib}
\usepackage[usenames]{color}
\usepackage{multirow}
\usepackage{listings}
\usepackage{textcomp}

\bibpunct{[}{]}{,}{n}{}{}

\def\denseitems{
  \itemsep1pt plus1pt minus1pt
  \parsep0pt plus0pt
  \parskip0pt\topsep0pt}

%\setpapersize{USletter}
%\setmarginsrb{2.54cm}{2.54cm}{2.54cm}{1.5cm}{0pt}{0mm}{0pt}{8mm}
%\setcounter{secnumdepth}{3}

\newcommand{\ie}{\textit{i.e.,} }
\newcommand{\eg}{\textit{e.g.,} }

\newcommand\opt[1]{}
\newcommand\find[1]{}

\newcommand{\xxx}[1]{{\color{red}\bf #1}}

\def\denseitems{
  \itemsep1pt plus1pt minus1pt
  \parsep0pt plus0pt
  \parskip0pt
  \topsep0pt
}
\leftmargini 1em
%  \leftmarginii-1em
%  \leftmarginiii-1em
%  \leftmarginvi-1em

\newenvironment{myquote}
  {\baselineskip=\SingleLine \begin{quote}
    {\setlength{\leftmargin}{2\parindent}}}
  {\end{quote} \baselineskip=\DoubleLine \vspace*{-\SingleLine}}

\def\bfcaption#1{\caption[#1]{\bf #1}}

\def\argmax{\mathop{\rm arg\,max}}

\def\denseitems{
  \itemsep1pt plus1pt minus1pt
  \parsep0pt plus0pt
  \parskip0pt\topsep0pt}

\newcommand{\ls}[1]
   {\dimen0=\fontdimen6\the\font 
    \lineskip=#1\dimen0
    \advance\lineskip.5\fontdimen5\the\font
    \advance\lineskip-\dimen0
    \lineskiplimit=.9\lineskip
    \baselineskip=\lineskip
    \advance\baselineskip\dimen0
    \normallineskip\lineskip
    \normallineskiplimit\lineskiplimit
    \normalbaselineskip\baselineskip
    \ignorespaces
   }

\newcommand{\crunchlist}{
        \setlength{\itemsep}{-0.05in}
        \setlength{\partopsep}{0in}
        \setlength{\topsep}{0in}}

\def\sqzhuge{\vspace{-14pt}}
\def\sqztiny{\vspace{-6pt}}

\def\definition#1#2{
\begin{smalldescription}
\item[Definition:] \underline{#1}\\
#2\hfill~$\Box$
\end{smalldescription}
%\vspace*{-5mm}
}

\def\definition#1{
\begin{smalldescription}
\item[Definition:]
#1\hfill~$\Box$
\end{smalldescription}
}

\def\ind#1{{\it #1}\index{#1}}

%% Alternative to itemize
\newenvironment{smallitemize}{
   \setlength{\topsep}{0pt}
   \setlength{\partopsep}{0pt}
   \setlength{\parskip}{0pt}
   \begin{itemize}
   \setlength{\leftmargin}{.2in}
   \setlength{\parsep}{0pt}
   \setlength{\parskip}{0pt}
   \setlength{\itemsep}{0pt}}{\end{itemize}}

%% Alternative to enumerate
\newenvironment{smallenumerate}{
   \setlength{\topsep}{0pt}
   \setlength{\partopsep}{0pt}
   \setlength{\parskip}{0pt}
   \begin{enumerate}
   \setlength{\leftmargin}{.2in}
   \setlength{\parsep}{0pt}
   \setlength{\parskip}{0pt}
   \setlength{\itemsep}{0pt}}{\end{enumerate}}

%% Alternative to description
\newenvironment{smalldescription}{
   \setlength{\topsep}{0pt}
   \setlength{\partopsep}{0pt}
   \setlength{\parskip}{0pt}
   \begin{description}
   \setlength{\leftmargin}{.2in}
   \setlength{\parsep}{0pt}
   \setlength{\parskip}{0pt}
   \setlength{\itemsep}{0pt}}{\end{description}}

%\long\gdef\TabPerf#1#2{%
%  #1 & #2 \\ \hline
%}

%% The [gray]{0} gives rise to the black color. For other shades of
%% gray, increase the number.
%\long\gdef\TabHead#1#2{%
%\multicolumn{1}{|>{\columncolor[gray]{0}}c|}{\textcolor{white}{\bf #1}} &
%\multicolumn{1}{>{\columncolor[gray]{0}}c|}{\textcolor{white}{\bf #2}} \\ \hline 
%}

%\long\gdef\TabPerf#1#2{%
%  #1 & #2 \\ \hline
%}

%% The [gray]{0} gives rise to the black color. For other shades of
%% gray, increase the number.
%\long\gdef\TabHead#1#2{%
%\multicolumn{1}{|>{\columncolor[gray]{0}}c|}{\textcolor{white}{\bf #1}} &
%\multicolumn{1}{>{\columncolor[gray]{0}}c|}{\textcolor{white}{\bf #2}} \\ \hline 
%}

%%% Local Variables: 
%%% mode: latex
%%% TeX-master: "paper"
%%% End: 


\makeatletter
\newcommand\tabcaption{\def\@captype{table}\caption}
\makeatother

\renewcommand\lstlistingname{\small{Listing}}

\newcommand{\lang}[1]{\texttt{\small #1}}
\newcommand{\subject}[1]{\texttt{\small #1}}
\newcommand{\mt}{\mathit}
\newcommand{\trex}{\textsc{TestEvol}}
\newcommand{\pass}{\mt{Pass}}
\newcommand{\fail}{\mt{Fail}}
\newcommand{\failce}{\mt{Fail}_{CE}}
\newcommand{\failre}{\mt{Fail}_{RE}}
\newcommand{\failae}{\mt{Fail}_{AE}}
\newcommand{\testfunc}[2]{\mt{Test(#1, #2)}}
\newcommand{\covfunc}[2]{\mt{Cov(#1, #2)}}
% \newcommand{\note}[1]{{\color{red}$\blacktriangleright$ \bf #1
%     $\blacktriangleleft$}}

\newcommand{\catrep}{\textsc{TestRep}}
\newcommand{\catref}{\textsc{TestModNotRep}}
\newcommand{\catdelaere}{\textsc{TestDel}$_\mt{(AE|RE)}$}
\newcommand{\catdelce}{\textsc{TestDel}$_\mt{(CE)}$}
\newcommand{\catdelp}{\textsc{TestDel}$_\mt{(P)}$}
\newcommand{\cataddaere}{\textsc{TestAdd}$_\mt{(AE|RE)}$}
\newcommand{\cataddce}{\textsc{TestAdd}$_\mt{(CE)}$}
\newcommand{\cataddp}{\textsc{TestAdd}$_\mt{(P)}$}

\newcommand{\smcatrep}{{\small \textsc{TestRep}}}
\newcommand{\smcatref}{{\small \textsc{TestRef}}}
\newcommand{\smcatdelaere}{{\small \textsc{TestDel}$_\mt{(AE|RE)}$}}
\newcommand{\smcatdelce}{{\small \textsc{TestDel}$_\mt{(CE)}$}}
\newcommand{\smcatdelp}{{\small \textsc{TestDel}$_\mt{(P)}$}}
\newcommand{\smcataddaere}{{\small \textsc{TestAdd}$_\mt{(AE|RE)}$}}
\newcommand{\smcataddce}{{\small \textsc{TestAdd}$_\mt{(CE)}$}}
\newcommand{\smcataddp}{{\small \textsc{TestAdd}$_\mt{(P)}$}}

\lstset{ %
upquote=true,
basicstyle=\tiny\sffamily,
showspaces=false,               % show spaces adding particular underscores
  showstringspaces=false,         % underline spaces within strings
  showtabs=false
}

%\linespread{0.97}
%\columnsep 0.11in
\clubpenalty = 10000
\widowpenalty = 10000
\displaywidowpenalty = 10000

\hypersetup{
colorlinks=true,
bookmarksnumbered=true,
pdftitle={Isolating Failure Causes through Test Case Generation},
pdfauthor={Jeremias R\"o\ss{}ler and Gordon Fraser and Andreas Zeller and Alessandro Orso},
pdfsubject={Automated Testing},
pdfkeywords={Automated Debugging, Diagnostics, Testing tools}
}

\hyphenation{Test-Evol}

\newcommand{\tool}{\textsc{TestEvol}\xspace}

\let\oldthebibliography=\thebibliography
\let\endoldthebibliography=\endthebibliography
\renewenvironment{thebibliography}[1]{%
  \begin{oldthebibliography}{#1}%
    \setlength{\parskip}{0ex}%
    \setlength{\itemsep}{0ex}%
  }%
  {%
  \end{oldthebibliography}%
}

%\toappear{}

\begin{document}

\title{TestEvol: A Tool for Analyzing Test-Suite Evolution}

\author{
\IEEEauthorblockN{Leandro Sales Pinto}
\IEEEauthorblockA{
Politecnico di Milano \\
pinto@elet.polimi.it}
\and
\IEEEauthorblockN{Saurabh Sinha}
\IEEEauthorblockA{
IBM Research -- India \\
saurabhsinha@in.ibm.com}
\and
\IEEEauthorblockN{Alessandro Orso}
\IEEEauthorblockA{
Georgia Institute of Technology \\
orso@cc.gatech.edu}
}

\maketitle

\pagestyle{empty}

% \note{This is a VERY rough draft that I put together by copying and
%   pasting from the FSE paper. I left some specific TODOs in, but the
%   paper really needs a major pass.}
% \todo{Restructure the content to make sure we comply with what is
%   requested from a demo paper: Tool-based demonstrations describe
%   novel aspects of early prototypes or mature tools. The tool
%   demonstrations must communicate clearly the following information to
%   the audience:\\
%   - the envisioned users;\\
%   - the software engineering challenge it proposes to address;\\
%   - the methodology it implies for its users; and\\
%   - the results of validation studies already conducted for mature tools, or the design of planned studies for early prototypes.}
% \todo{Reduce the description of the approach and extend the
%   description of the implementation. Leandro, can you add some details
%   there?}
% \todo{Add snapshots.}

\begin{abstract}
  Test suites, once created, rarely remain static. Just like the
  application they are testing, they evolve throughout their lifetime.
  Test obsolescence is probably the most known reason for test-suite
  evolution---test cases cease to work because of changes in the code
  and must be suitably repaired. There are several reasons why it is
  important to achieve a thorough understanding of how test cases
  evolve in practice. In particular, researchers who develop automated
  test-repair techniques---an increasingly active research area---can
  use such understanding to develop more effective repair techniques
  that are applicable in real-world scenarios. More generally,
  analyzing test-suite evolution can help testers better understand
  how test cases are modified during maintenance and improve the test
  evolution process, an extremely time consuming activity for any non
  trivial test suite.  Unfortunately, there are no existing tools that
  facilitate investigation of test-suite evolution. To tackle this
  problem, we developed \tool, a tool that enables the systematic
  study of test-suite evolution for Java programs and JUnit test
  cases.  This demonstration presents \tool and demonstrates its
  usefulness and practical applicability by showing how \tool\ can be
  used on real-world software and unit test suites.

  \noindent
  Demo video at {\small \url{http://www.cc.gatech.edu/~orso/software/testevol/}}

\end{abstract}

% A category with the (minimum) three required fields
%\category{H.4}{Information Systems Applications}{Miscellaneous}
%A category including the fourth, optional field follows...
%\category{D.2.8}{Software Engineering}{Metrics}[complexity measures, performance measures]

%\terms{Theory}

%\keywords{ACM proceedings, \LaTeX, text tagging}

\section{Motivation and Overview}
\label{sec:intro}

Test suites are not static entities: they constantly evolve along with
the application they test. For instance, new tests can be added to
test new functionality and existing tests can be refactored, repaired,
or deleted.  In particular, changes in the application can break test
cases---in some cases, even a small change in the application code can
affect a large number of tests.

A broken test, if it covers a valid functionality, should ideally be
repaired. Alternatively, if the repair is unduly complex to perform or
if the test was designed to cover a functionality that no longer
exists in the application, the test should be removed from the test
suite. To illustrate with an example,
Figure~\ref{fig:pmd-javatokenizertest} shows two versions of a unit
test case from the test suite of \subject{PMD}, one of the programs we
analyzed in our previous work~\cite{pinto12}. A change in
\subject{PMD}'s API broke the original version of the test case, which
had to be fixed by adding a call to method
\lang{SourceCode.readSource} and removing one parameter from the call
to method \lang{Tokenizer.tokenize} (lines 6 and 7 in
Figure~\ref{fig:pmd-javatokenizertest}(b), respectively).

\begin{figure}[t]
\lstinputlisting[language=java,
captionpos=t,frame=tb,stepnumber=1,numbers=left,numbersep=5pt]{code/a.java}
\vspace*{-4pt}
\centerline{(a)}
\lstinputlisting[language=java,
captionpos=t,frame=tb,stepnumber=1,numbers=left,numbersep=5pt]{code/b.java}
\vspace*{-4pt}
\centerline{(b)}
\vspace*{-8pt}
\caption{Two versions of a test case from \subject{PMD}'s unit test
  suite: (a) version~1.4, broken, and (b) version~1.6, repaired.}
\vspace*{-12pt}
\label{fig:pmd-javatokenizertest}
\end{figure}

Because test repair can be an expensive activity, automating it---even
if only partially---could save a considerable amount of resources
during maintenance. This is the motivation behind the development of
automated test-repair techniques, such as the ones targeted at unit
test cases~\cite{Daniel:2009, Daniel:2010, Mirzaaghaei:2012} and those
focused on GUI (or system) test cases~\cite{Choudhary:2011,
  Grechanik:2009, Huang:2010, Memon:2008}.

We believe that, to develop effective techniques for assisting manual
test repair, we must first understand how test suites evolve in
practice. That is, we must understand when and how tests are created,
removed, and modified. This is a necessary, preliminary step because
it can (1) provide evidence that test cases do get repaired, (2)
support the hypothesis that test repairs can be (at least partially)
automated, and (3) suitably direct research efforts.  Without such
understanding, we risk to develop techniques that may not be generally
applicable and may not perform the kind of repairs that are actually
needed in real-world software systems.

More generally, the ability to analyze how test suites evolve can be
beneficial for developers, as it can help them better understand the
cost and tradeoffs of test evolution. Similarly, project managers can
use information about test-suite evolution to get insight into the
test maintenance process (\eg how often tests are deleted and cause
oss of coverage, how often tests must be adapted because of changes in
the system's API) and ultimately improve it.

Until recently there were no empirical studies in the literature that
investigated how unit test suites evolve and tools that could support
such studies. To address this issue, we defined a technique that
combines various static- and dynamic-analysis techniques to compute
the differences between the test suites associated with two versions
of a program and categorize such changes along two dimensions: the
static differences between the tests in the two test suites and the
behavioral differences between such tests~\cite{pinto12}. Applying our
technique to a number of real-world programs, we were able to discover
several important aspects of test evolution. For example, we found
evidence that, although test repairs are a relatively small fraction
of the activities performed during test evolution, they are indeed
relevant.  We also found that repair techniques that just focus on
oracles (\ie assertions) are likely to be inapplicable in many cases,
that test cases are rarely removed because they are difficult to fix,
but rather because they have become obsolete, and that test cases are
not only added to check bug fixes and test new functionality, as
expected, but also to validate changes in the code.

This demonstration presents a tool that implements our technique and
that we developed to allow other researchers and practitioners to
perform studies such as we the ones we just summarized. The tool,
which is called \tool, enables a systematic study of test-suite
evolution.  Given two versions of a program and corresponding test
suites, \tool\ automatically computes differences in the behavior of
the test suites on the two program versions, classifies the actual
repairs performed between the two versions, and computes the coverage
attained by the tests on the versions.

After presenting \tool's features and technical details, the
demonstration shows how the tool can be used to study how test suites
evolved over the years for a software project.  To do so, we show
examples of application of \tool\ to a set of real-world open-source
software systems.  Specifically, we show how to use \tool\ to
investigate relevant questions on test evolution, such as the
following ones: What types of test-suite changes occur in practice and
with what frequency?  How often do test repairs require complex
modifications of the tests?  Why are tests deleted and added? Overall,
we show how \tool\ is a useful and practically applicable tool for
researchers in the area of test repair (and test evolution in
general), developers and testers who want to better understand their
software maintenance process.

\section{The \tool\ Tool}
\label{sec:test-evolution}

In this section we firs summarize our technique, implemented in the
\tool\ tool, and then discuss the main characteristics of the tool
implementation. A detailed description of the approach can be found
in~\cite{pinto12}.

\subsection{Underlying Technique}

Before describing the characteristics of our technique, we introduce
some necessary terminology.  A \textit{system} $S = (P, T)$ consists
of a program $P$ and a test suite $T$.  A \textit{test suite} $T =
\{t_1, t_2, \ldots, t_n\}$ consists of a set of unit test cases.
% A unit test case $t$ is a method (or function) and it is identified
% by its name plus the name of the class it is associated with.
$\testfunc{P}{t}$ is a function that executes test case $t$ on program
$P$ and returns the outcome of the test execution.  A \textit{test
  outcome} can be of one of four types:

\begin{itemize}\denseitems

\item $\pass$: The execution of $P$ against $t$ succeeds.

\item $\failce$: The execution of $P$ against $t$ fails because a
  class or method accessed in $t$ does not exist in
  $P$.\footnote{\small These failures can obviously be detected at
    compile-time. For the sake of consistency in the discussion,
    however, we consider such cases to be detected at runtime via
    ``class not found'' or ``no such method'' exceptions. In fact, our
    \tool{} tool detects such failures at runtime by executing the
    tests compiled using the previous version of $P$ on $P$.}

\item $\failre$: The execution of $P$ against $t$ fails due to an
  uncaught runtime exception (\eg a ``null pointer'' exception).

\item $\failae$: The execution of $P$ against $t$ fails due to an
  assertion violation.

\end{itemize}

We use the generic term $\fail$ to refer to failures for which the
distinction among the above three types of failures is unnecessary.

$\covfunc{P}{t}$ is a function that instruments program $P$, executes
test case $t$ on $P$, and returns the set of all statements in $P$
covered by $t$. $\covfunc{P}{T}$ returns the cumulative coverage
achieved on $P$ by all the tests in test suite $T$.

% $\covfunc{P}{t}$ is a function that instruments program $P$, executes
% test case $t$ on $P$, and returns the number of branches in $P$
% covered by $t$. $\covfunc{P}{T}$ returns the cumulative coverage
% attained on $P$ by all the tests in test suite $T$.

%% Unit tests are run very frequently---often from the IDE, soon after
%% the code is changed, or even while the code is being changed. In a
%% typical testing process, developers may run the test suite after
%% editing the code. In the run, some tests may fail, whereas others
%% would pass. The developer would want to fix the broken tests online to
%% ensure that all tests pass.

Given a system $S = (P, T)$, a modified version of $S$, $S'=(P', T')$,
and a test case $t$ in $T \cup T'$, there are three possible scenarios
to consider: (1) $t$ exists in $T$ and $T'$, (2) $t$ exists in $T$ but
not $T'$ (\ie $t$ was removed from the test suite), and (3) $t$ exists
in $T'$ but not in $T$ (\ie $t$ was added to the test suite). These
scenarios can be further classified based on the behavior of $t$ in
$S$ and $S'$, as summarized in Figure~\ref{fig:study-design} and
discussed in the rest of this section.

\paragraph*{\textbf{Test Modifications}}
\label{sec:test-mod}

Figure~\ref{fig:study-design}(a) illustrates the scenario in which $t$
is present in the test suites for both the old and the new versions of
the system. To study different cases, we consider whether $t$ is
modified (to $t'$) and, if so, whether the behaviors of $t$ and $t'$
differ. (We have not considered the cases in which $t$ is not modified
because they are irrelevant for studying test evolution.)  For
\textit{behavioral differences}, there are two cases, shown in the two
rows of the table: either $t$ fails on $P'$ and $t'$ passes on $P'$ or
both $t$ and $t'$ pass on $P'$.

\begin{figure}[t]
\centering
\footnotesize
\tabcolsep=3pt
\textbf{(a) Test $t$ exists in $S$ and $S'$ and is modified}
\\ [2pt]
\begin{tabular}{|l||l|}
\hline
%\multicolumn{1}{|c||}{} & \multicolumn{1}{|c}{} &
%\multicolumn{1}{|c|}{$t$ is not} \\
%\multicolumn{1}{|c||}{} & \multicolumn{1}{|c|}{$t$ is modified}\\
%\hline \hline
$\testfunc{P'}{t} = \fail \; \wedge$ &
$t$ is repaired \\
$\testfunc{P'}{t'} = \pass$ &
\multicolumn{1}{r|}{\scriptsize [\catrep{}]}\\
\hline
\multirow{3}{28mm}{$\testfunc{P'}{t} = \pass \; \wedge \testfunc{P'}{t'} = \pass$} &
$t$ is refactored, updated to test a different \\
& scenario, or is made more/less discriminating \\
& \multicolumn{1}{r|}{\scriptsize [\catref{}]} \\
\hline
\end{tabular}
\\ [8pt]
%
\textbf{(b) Test $t$ is removed in $S'$}
\\ [2pt]
\begin{tabular}{|l||l|}
\hline
\multirow{2}{*}{$\testfunc{P'}{t} = \failre \, | \, \failae$} &
$t$ is too difficult to fix \\
& \multicolumn{1}{r|}{\scriptsize [\catdelaere{}]} \\
\hline
\multirow{2}{*}{$\testfunc{P'}{t} = \failce$} & $t$ is obsolete or is
too difficult to fix \\
& \multicolumn{1}{r|}{\scriptsize [\catdelce{}]} \\
\hline
\multirow{2}{*}{$\testfunc{P'}{t} = \pass$} & $t$ is redundant \\
& \multicolumn{1}{r|}{\scriptsize [\catdelp{}]} \\
\hline 
\end{tabular}
\\ [8pt]
%
\textbf{(c) Test $t'$ is added in $S'$}
\\ [2pt]
\begin{tabular}{|l||l|}
  \hline
  \multirow{2}{*}{$\testfunc{P}{t'} = \failre \, | \, \failae$} &
  $t'$ is added to validate a bug fix \\
  & \multicolumn{1}{r|}{\scriptsize [\cataddaere{}]} \\
  \hline
  \multirow{3}{*}{$\testfunc{P}{t'} = \failce$} &
  $t'$ is added to test a new functionality \\
  & or a code refactoring \\
  & \multicolumn{1}{r|}{\scriptsize [\cataddce{}]} \\
  \hline
  \multirow{3}{*}{$\testfunc{P}{t'} = \pass$} &
  %% TODO: Check: $t'$ is added to test a new functionality \\
  $t'$ is added to test an existing feature\\
  & or for coverage-based augmentation \\
  & \multicolumn{1}{r|}{\scriptsize [\cataddp{}]} \\
  \hline
\end{tabular}
\caption{Scenarios considered in our investigation. Given two system
  versions $S = (P, T)$ and $S' = (P', T')$, the three scenarios are:
  (a) $t$ exists in $T$ and $T'$ and is modified, (b) $t$ exists in
  $T$ but not in $T'$, (c) $t'$ exists in $T'$ but not in $T$.}
\vspace*{-5pt}
\label{fig:study-design}
\end{figure}

\paragraph{Category \catrep{} (Repaired Tests)}
\label{sec:category-rep}

The \catrep{} category corresponds to the case where $t$ is repaired
so that, after the modifications, it passes on $P'.$ As discussed in
the Introduction, Figure~\ref{fig:pmd-javatokenizertest} shows an
example of such a test repair.

%% The code fragments in Listings~\ref{c1-old} and~\ref{c1-new} present
%% another example of repair that involves a simpler code modification
%% than the one in Figure~\ref{fig:pmd-javatokenizertest}. The example is
%% taken from \subject{Gson}, one of the programs used in our demo, and
%% involves a test case from \subject{Gson} version~2.0 that was fixed in
%% the subsequent version~2.1. Test \lang{testNullField} had to be fixed
%% because constructor \lang{FieldAttributes(Class<?> declClazz, Field
%%   f)} from version~2.0 was modified in version~2.1 to take only one
%% parameter (of type \lang{Field}).
%% %% was no longer available on version 2.1. In the new version of the
%% %% \lang{FieldAttributes} class, the constructor receives only one
%% %% parameter of type \lang{Field}.

%% \vspace*{-10pt}
%% \lstinputlisting[language=java,label=c1-old,caption=\small{Unit test
%%   for class FieldAttributes} (Gson v2.0),
%% captionpos=t,frame=tb]{code/OldTestRep.java}

%% \vspace*{-5pt}
%% \lstinputlisting[language=java,label=c1-new,caption=\small{Unit test
%%   for class FieldAttributes} (Gson v2.1),
%% captionpos=t,belowcaptionskip=4pt,frame=tb]{code/TestRep.java}

For this category, we wish to study the types of modifications that
are made to $t$. A test repair may involve changing the sequence of
method calls, assertions, data values, or control flow. Based on our
experience, for \textit{method-call sequence changes}, we consider
five types of modifications:

\begin{enumerate}
\denseitems

\item \textit{Method call added}: a new method call is added.

\item \textit{Method call deleted}: an existing method call is
  removed.

\item \textit{Method parameter added}: a method call is modified such
  that new parameters are added.

\item \textit{Method parameter deleted}: a method call is modified
  such that existing parameters are deleted.

\item \textit{Method parameter modified}: a method call is modified
  via changes in the values of its actual parameters.

\end{enumerate}

A test repair may involve multiple such changes. For example, the
repair shown in Figure~\ref{fig:pmd-javatokenizertest} involves the
addition of a method call (line~6) and the deletion of a method
parameter (line~7).
%% The repair illustrated in Listings~\ref{c1-old} and~\ref{c1-new} also
%% involves the deletion of a method parameter.
%
For \textit{assertion changes}, we consider cases in which an
assertion is added, an assertion is deleted, the expected value of an
assertion is modified, or the assertion is modified but the expected
value is unchanged.
%
Finally, we currently group together \textit{data-value changes} and
\textit{control-flow changes}.

The rationale underlying our classification scheme is that different
classes of changes may require different types of repair analyses.
Although this is not necessarily the case, at least the search
strategy for candidate repairs would differ for the different
categories of changes. Consider the case of method-parameter deletion,
for instance, for which one could attempt a repair by simply deleting
some of the actual parameters. Whether this repair would work depends
on the situation. For the code in
Figure~\ref{fig:pmd-javatokenizertest}, for example, it would not work
because the deletion of one of the parameters in the call to
\lang{tokenize()} is insufficient by itself to fix the test---a new
method call (to \lang{readSource()}) has to be added as well for the
test to work correctly.
%% Similarly, for the case of method-parameter addition, one could
%% conceivably attempt straightforward fixes by constructing equivalence
%% classes for the new parameters and selecting a value from each
%% equivalence class (\eg positive, negative, and zero values for an
%% integer parameter).  In our study, we found cases where such an
%% approach would in fact repair broken tests. In this case too, however,
%% such a solution is in general not enough.

\paragraph{Category \catref{} (Refactored Tests)}
\label{sec:category-c2}

The \catref{} category captures scenarios in which a test $t$ is
modified in $S'$ even though $t$ passes on $P'$.
%% Listings~\ref{c2-old} and~\ref{c2-new} show an example of one such
%% change from \subject{Commons Math}. Unit test \lang{testDistance} was
%% refactored to invoke method \lang{sqrt()} in class \lang{FastMath}, a
%% newly added class in the new release, instead of the same method in
%% \lang{java.lang.Math}.  

%% \vspace{-5pt}
%% \lstinputlisting[language=java,label=c2-old,caption=\small{Unit test
%%   from class Vector3DTest} (Commons Math v2.1),
%% captionpos=t,frame=tb]{code/OldTestRef.java}

%% \vspace{-5pt}
%% \lstinputlisting[language=java,label=c2-new,caption=\small{Unit test
%%   from class Vector3DTest} (Commons Math v2.2),
%% captionpos=t,frame=tb]{code/TestRef.java}

\paragraph*{\textbf{Test Deletions}}
\label{sec:test-ref}

Figure~\ref{fig:study-design}(b) illustrates the scenario in which a
test $t$ is deleted. To study the reasons for this, we examine the
behavior of $t$ on the new program version $P'$ and consider three
types of behaviors.

\paragraph{Category \catdelaere{} (Hard-To-Fix Tests)}
\label{sec:category-delaere}

%%COMMENT PLEASE READ ME: I guess here we should focus more
%%in the cases that tests are removed because they have become
%%obsolete. Actually, in the examples I was not able to find one
%%case in which it was removed because it was too difficult to
%% be fixed. It was always obsolete tests.

This category includes tests that fail on $P'$ with a runtime
exception or an assertion violation. These may be instances where the
tests should have been fixed, as the functionality that they test in
$P$ still exists in $P'$, but the tests were discarded instead. One
plausible hypothesis is that tests in this category involve repairs of
undue complexity, for which the investigation of new repair techniques
to aid the developer might be particularly useful.
%% We performed a preliminary investigation of this hypothesis by
%% manually examining ten randomly selected tests in this category and
%% found that all the examined tests were in fact obsolete.

\paragraph{Category \catdelce (Obsolete Tests)}
\label{sec:category-delce}

A test that fails with a compilation error on the new program version
is obsolete because of API changes.
%
%% Listing~\ref{c4-example} illustrates a test in this category taken
%% from \subject{JodaTime}.  This test was deleted because the tested
%% method \lang{Chronology.getBuddhist()} was removed in the subsequent
%% version of \subject{JodaTime}.
%
%% % \vspace{-4pt}
%% \lstinputlisting[language=java,label=c4-example,caption=\small{Unit
%%   test from class TestChronology} (JodaTime v2.0),frame=tb]
%% {code/TestDel-Ce.java}
%% %This method was
%% %already marked as deprecated in the previous version.
%
Although for this category of deletion too, one could postulate that
the tests were removed because they were too difficult to fix, we
believe this not to be the case in most practical occurrences.
Instead, the more likely explanation is that the tests were removed
simply because the tested methods were no longer present.
%% Also in this case, we investigated our hypothesis by manually
%% examining ten randomly selected cases. Indeed, our manual
%% investigation confirmed that, for the cases we analyzed, the tested
%% functionality was either removed or provided through alternative
%% methods, which requires the development of new tests rather than fixes
%% to the existing ones.

\paragraph{Category \catdelp: (Redundant Tests)}
\label{sec:category-delp}

This category includes tests that are removed even though they pass on
$P'$. These tests would typically be redundant, by some measure of
redundancy (\eg code-coverage based). In some cases, a test may be
replaced with a different one.

%% Listing~\ref{c5-example} illustrates an example of one such test,
%% taken from \subject{Commons Lang}. In the new version, this test was
%% replaced by a more sophisticated one, shown in
%% Listing~\ref{c5.2-example}.

%% \begin{figure}[h]
%% \lstinputlisting[language=java,label=c5-example,caption=\small{Unit
%%   test from class MatrixIndexExceptionTest} (Commons Lang
%% v2.1),frame=tb] {code/TestDel-P.java}
%% \vspace{4pt}
%% \end{figure}

%% \lstinputlisting[language=java,label=c5.2-example,caption=\small{Unit
%%   test implemented to replace the removed one} (Commons Lang
%% v2.2),frame=tb] {code/TestDel-PII.java}

\paragraph*{\textbf{Test Additions}}
\label{sec:test-add}

Figure~\ref{fig:study-design}(c) illustrates the cases of test-suite
augmentation, where a new test $t'$ is added to the test suite. The
behavior of $t'$ on the old program can indicate the reason why it may
have been added.

\paragraph{Category \cataddaere: (Bug-Fix Tests)}
\label{sec:category-addaere}

This category includes added tests that fail on $P$ with a runtime
exception or an assertion violation. In this case, the functionality
that the added test $t'$ was designed to test exists in $P$ but is not
working as expected (most likely because of a fault). The program
modifications between $P$ and $P'$ would ostensibly have been made to
fix the fault, which causes $t'$ to pass on $P'$. Thus, $t'$ is added
to the test suite to validate the bug fix.

%% Listing~\ref{c6-example} illustrates a test that fits this profile, as
%% it was added to \subject{Commons Math} version 2.1 to validate a bug
%% fix ({\small
%%   \url{https://issues.apache.org/jira/browse/MATH-391}}). In the
%% previous version of \lang{ArrayRealVector}, the creation of a
%% zero-length vector results in a runtime exception.

%% % \begin{figure}[h]
%% \lstinputlisting[language=java,label=c6-example,caption=\small{\mbox{Unit
%%   test from class ArrayRealVectorTest (Commons Math v2.2)}},
%% captionpos=t,frame=tb]{code/TestAdd-AeRe.java}
%% % \end{figure}

\paragraph{Category \cataddce (New-Features Tests)}
\label{sec:category-addce}

The added tests in this category fail on $P$ with a compilation error,
which indicates that the API accessed by the tests does not exist in
$P$. Thus, the added test $t'$ is created to test new code in $P'$,
where the code could have been added as part of a refactoring or, more
likely, to add new functionality.

%% Listing~\ref{c7-example} illustrates a test from \subject{JFreeChart}
%% that covers a new functionality: method \lang{getMinY()} did not exist
%% in previous versions of class \lang{TimeSeries}.

%% \lstinputlisting[language=java,label=c7-example,caption=\small{Unit
%%   test from class TimeSeriesTests} (JFreeChart v1.0.14),
%% captionpos=t,frame=tb]{code/TestAdd-Ce.java}

\paragraph{Category \cataddp (Coverage-Augmentation Tests)}
\label{sec:category-addp}

This category considers cases where the added test $t'$ passes on
$P$. Clearly, $t'$ would have been a valid test in the old system as
well. One would expect that the addition of $t'$ increases program
coverage (\ie $\covfunc{P'}{T'} \supset \covfunc{P'}{T' -
  \{t'\}}$). Moreover, if $t'$ covers different statements in $P$ and
$P'$ (assuming that there is a way of matching statements between $P$
and $P'$), the plausible explanation is that $t'$ was added to test
the changes made between $P$ and $P'$. However, if $t'$ covers the
same statements in both program versions, it would have been added
purely to increase code coverage (and not to test any added or
modified code).

%% Listing~\ref{c8-example} illustrates an added test case that increases
%% code coverage: the new test method \lang{testFindRangeBounds} covers
%% the case where the parameter of method \lang{findRangeBounds()} is
%% null.

%% \lstinputlisting[language=java,label=c8-example,caption=\small{Unit test
%% from class XYErrorRendererTests} (JFreeChart v1.0.14),frame=tb]
%% {code/TestAdd-P.java}

\section{Implementation}
\label{sec:implementation}

% \todo{Add the discussion of clone analysis.}
% \todo{In general, add more details.}
% \todo{Discuss the availability of the tool.}

In our demonstration, we show an implementation of our technique,
\tool, that works on Java programs and JUnit test suites ({\small
  \url{http://www.junit.org/}}). We chose Java and JUnit because the
former is a widely used language, and the latter is the de-facto
standard unit-testing framework for Java.  \tool\ analyzes a sequence
of versions of a software system, where each version can be an actual
release or an internal build and consists of application code and test
code.

\tool\ is implemented as a Java based web application that runs on
Apache Tomcat ({\small \url{http://tomcat.apache.org/}}). Although the
tool can be installed locally, users who want to get familiar with
\tool\ without having to install it can also access it remotely at
{\small \url{http://cheetah.cc.gt.atl.ga.us:8081/testevol/}},
specifying user ``icse'' and password ``icse2013''.

Due to space limitations, we can only provide a few examples of
\tool's graphical interface. (The companion demo video illustrates the
tool interface in full detail.) Figure~\ref{fig:summary} shows the
summary report generated by \tool\ for two pairs of versions of
project \subject{google-gson}. The report shows the number of tests,
both total and per version, that fall in each of the eight categories
defined in our approach.  By clicking on a pair of versions, the user
can obtain detailed report for a specific pair of versions. To
illustrate, Figure~\ref{fig:versiondetails} shows the detailed report
for versions $v.1.1$--$v.1.2$ and a specific category of differences
(\catrep). By further clicking on a test case in the list, users can
also inspect, in the case of modified tests, the actual differences
between the two versions of that test, as shown in
Figure~\ref{fig:summary} for test case \subject{TypeInfoTest}.

\begin{figure}[t]
	\centering
	\includegraphics[width=\columnwidth]{1-summary}
        \vspace*{-20pt}
	\caption{Summary report for two pairs of versions of a
          project.}
        \vspace*{-8pt}
	\label{fig:summary}
\end{figure}

\begin{figure}[t]
	\centering
	\includegraphics[width=\columnwidth]{2-versiondetails}
        \vspace*{-20pt}
	\caption{Detailed report for a pair of versions.}
        \vspace*{-8pt}
	\label{fig:versiondetails}
\end{figure}

\begin{figure}[t]
	\centering
	\includegraphics[width=\columnwidth]{3-testdetails}
        \vspace*{-20pt}
	\caption{Differences between two versions of a test case.}
        \vspace*{-8pt}
	\label{fig:testdetails}
\end{figure}

In terms of design, \tool\ consists of five components, as illustrated
in Figure~\ref{fig:trex}.
%
The \emph{compiler} component builds each system version and creates
two jar files, one containing the application classes and the other
containing the test classes.
%
The \emph{test-execution engine} analyzes each pair of system versions
$(S, S')$, where $S = (P, T)$ and $S' = (P', T')$.  First, it executes
$T$ on program $P$ and $T'$ on program $P'$ (\ie it runs the tests on
the respective program versions).  Then, it executes $T'$ on $P$ and
$T$ on $P'$. For each execution, it records the test outcome: $\pass,
\failce, \failae, or \failre$.
%
The \emph{differencing component} compares $T$ and $T'$ to identify
modified, deleted, and added tests. This component is implemented
using the \textsc{wala} analysis infrastructure for Java ({\small
  \url{wala.sourceforge.net}}).

The test outcomes, collected by the test-execution engine, and the
test-suite changes, computed by the differencing component, are then
passed to the \emph{test classifier}, which analyzes the information
about test outcomes and test updates to classify each update into one
of the eight test-evolution categories presented in
Section~\ref{sec:test-evolution}. For each pair consisting of a broken
test case and its repaired version, the test classifier also compares
the test cases to identify the types of repair changes---different
types of method-sequence changes and assertion changes---as discussed
in Section~\ref{sec:category-rep}; this analysis is also implemented
using \textsc{wala}.

\tool\ performs a further step for the test cases in categories
\catdelp{}and \cataddp{}. For these tests, the test classifier
leverages the \emph{coverage analyzer} to compute the branch coverage
achieved by each test; this facilitates the investigation of whether
the deleted or added tests cause any variations in the coverage
attained by the new test suite.

\begin{figure}
	\centering
	\includegraphics[width=\columnwidth]{architecture}
        \vspace*{-20pt}
	\caption{High-level design of \tool.}
        \vspace*{-8pt}
	\label{fig:trex}
\end{figure}

\vspace*{-8pt}
\section{Conclusion}
\label{sec:summary}

Test suites evolve throughout their lifetime and change with the
applications under test. Adapting existing test cases manually,
however, can be extremely tedious, especially for large test suites,
which has motivated the recent development of automated test-repair
techniques (\eg \cite{Daniel:2009, Daniel:2010, Mirzaaghaei:2012}).
To support the development of more effective repair techniques, we
developed \tool, a tool for systematic investigation of test-suite
evolution that can provide its users with a comprehensive
understanding of how test cases evolve. In this demonstration, we
present \tool, its main features, and its technical characteristics.
We also demonstrate how \tool\ can be run on real-world software
systems and produce a wealth of useful information, such as the types
of test-suite changes that occur, the frequency of the changes, and
why tests are deleted and added.  Overall, we show how \tool\ is a
useful tool not only for researchers working in the area of test
repair, and test evolution in general, but also for developers and
testers who wish to understand their software systems better.

% \section*{Acknowledgments}

% This work was supported in part by NSF awards CCF-0916605 and
% CCF-0964647 to Georgia Tech, and by funding from IBM Research and
% Microsoft Research.

\bibliographystyle{abbrv}
{\footnotesize
\bibliography{paper}
}

\end{document}

